\documentclass[10pt,a4paper]{article}
\usepackage[english]{babel}
\usepackage[utf8]{inputenc}
\usepackage{url}
\usepackage{csquotes}
\usepackage{amsmath}
\usepackage{amssymb}
\usepackage{isabelle,isabellesym}

\usepackage{color}

\usepackage[top=3cm,bottom=4.5cm]{geometry}

\definecolor{keyword}{RGB}{0,153,102}
\definecolor{command}{RGB}{0,102,153}
\isabellestyle{tt}
\renewcommand{\isacommand}[1]{\textcolor{command}{\textbf{#1}}}
\renewcommand{\isakeyword}[1]{\textcolor{keyword}{\textbf{#1}}}

% this should be the last package used
\usepackage{pdfsetup}

% urls in roman style, theory text in isabelle-similar-similar type-writer
\urlstyle{rm}
\isabellestyle{tt}

\title{ \textbf{Coupled Similarity} \\ \Large and How to Compute It }
\author{ Benjamin Bisping%
  \footnote{Technische Universit\"at Berlin, Germany,
    \href{mailto:benjamin.bisping@tu-berlin.de}{benjamin.bisping@tu-berlin.de}.} }
%
\date{\today}

\begin{document}

\maketitle

\begin{abstract}
\noindent
This theory surveys a range of definitions of \emph{coupled similarity} from the literature,
and proves properties relevant for algorithms computing coupled similarity relations.

Coupled similarity is a notion of equivalence for systems with internal actions.
It has outstanding applications in contexts where internal choices must transparently be
distributed in time or space, for example, in process calculi encodings or in action refinements.

We show how the preexisting definitions coincide and that they can be reformulated using
\emph{coupled delay simulations}. Our key contribution is to verify a polynomial-time coinductive
fixed-point algorithm computing the coupled simulation preorder and to characterize the
coupled simulation preorder by a simple game. Our proofs also support the conclusion that
computing coupled similarity is at least as complex as computing weak similarity.
\end{abstract}

\tableofcontents

\section{Introduction}

This theory accompanies Benjamin Bisping and Uwe Nestmann's TACAS 2019 paper \cite{bn2019coupledsimTacas}
and Benjamin Bisping's master's thesis ``Computing Coupled Similarity'' \cite{bisping2018coupledsim},
which can be found on \url{https://coupledsim.bbisping.de/bisping_computingCoupledSimilarity_thesis.pdf}.


% include generated text of all theories
\input{session}

\phantomsection
\addcontentsline{toc}{section}{References}
\bibliographystyle{splncs04}
\bibliography{root}

\end{document}
